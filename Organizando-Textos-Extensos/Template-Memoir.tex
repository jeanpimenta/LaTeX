%%%%%%%%%%%%%%%%%%%%%%%%%%%%%%%%%%%%%%%%%%%%%%%%%%%%
%%% Template criado por Jean Pimenta
%%%                v0.1
%%%      youtube.com/jeanpimenta
%%%     www.jeanpimenta.com/latex/
%%%     
%%%
%%% Todo o projeto foi compilado com XeLaTeX.
%%%
%%% Chamados de pacotes e definições se encontram
%%% na pasta preâmbulo dentro do projeto.
%%%
%%%%%%%%%%%%%%%%%%%%%%%%%%%%%%%%%%%%%%%%%%%%%%%%%%%%

\documentclass[12pt,showtrims]{memoir}

\usepackage{import}

% input: adiciona as linhas do arquivo chamado dentro do arquivo 
% onde é escrito. Não é possível chamar outro arquivo dentro do
% qual já o chama via input.
%
% include: o mesmo que input, mas adiciona um \clearpage antes
% de importar as linhas do arquivo chamado. Há relatos de pro-
% blemas com importação de referências ao se usar include.
%
% import: com uso do pacote import. Chama as linhas do arquivo
% e pode concatenar vários arquivos dentro de outros com o
% comando subimport
%
%%%%%%%%%%%%%%%%%%%%%%%%%%%%%%%%%%%%%%%%%%%%%%%%%%%%%
%%% Pequena introdução sobre comandos para dimensões
%%% do papel, margens, etc., pelo memoir
%%%
%%% Para mais explicação e informações de comandos
%%% leia a documentação do memoir: memman.pdf
%%% http://texdoc.net/texmf-dist/doc/latex/memoir/memman.pdf
%%%
%%% \setstocksize{Altura}{Largura} %Define dimensões do papel
%%%
%%% \settrimmedsize{Altura}{Largura}{Razão} 
%%% Define a área a ser
%%% utilizada para o texto, figuras. Caso este comando não 
%%% esteja definido, memoir utilizará toda a dimensão do papel.
%%% Apenas um ou dois valores são necessários. Os outros, se
%%% não especificados, devem ser *.
%%%
%%% \setpagecc{\paperheight}{\paperwidth}{*} %
%%% O comando, deste jeito centralizará a área do texto no papel
%%%
%%% \setlrmarginsandblock{Marg. Dir.}{Marg. Esq.}{Razão}
%%% Define as margens direta e esquerda.
%%% Apenas um ou dois valores são necessários. Os outros, se
%%% não especificados, devem ser *
%%%
%%% \setulmarginsandblock{Margem Sup.}{Margem Inf.}{Razão}
%%% Define as margens superio e inferior.
%%% Apenas um ou dois valores são necessários. Os outros, se
%%% não especificados, devem ser *.
%%%
%%% \checkandfixthelayout 
%%% Este comando calcula corretamente os parâmetros pedidos 
%%% e diz se há erro ou não. É sempre necessário ao final dos 
%%% comandos para setar as dimensões do papel desta maneira, 
%%% com memoir.
%%%
%%%%%%%%%%%%%%%%%%%%%%%%%%%%%%%%%%%%%%%%%%%%%%%%%%%%


\def\PapelA{%
% Uncomment to manually set the stock size and override the setting in \documentclass. 
%\setstocksize{24cm}{17cm}
% Change the trimmed area size or comment out this line entirely to fit the content to the paper size without trimming.
\settrimmedsize{24cm}{17cm}{*}
% The first bracket specifies the spine margin, the second the edge margin and the third the ratio of the spine to the edge. Only one or two values are required and the remaining one(s) can be a star (*) to specify it is not needed.
\setlrmarginsandblock{22mm}{*}{0.9}
% The first bracket specifies the upper margin, the second the lower margin and the third the ratio of the upper to the lower. Only one or two values are required and the remaining one(s) can be a star (*) to specify it is not needed.
\setulmarginsandblock{26mm}{20mm}{*}


% The size of marginal notes, the three values in curly brackets are \marginparsep, \marginparwidth and \marginparpush.
\setmarginnotes{17pt}{51pt}{\onelineskip}
% Sets the space available for the header and footer
\setheadfoot{\onelineskip}{2\onelineskip}
% Sets the spacing above and below the header
\setheaderspaces{*}{2\onelineskip}{*}


% Sets the spacing above the trimmed area, i.e. moved the trimmed area down the page if positive.
\setlength{\trimtop}{0pt}


% Comment the two lines below to reverse the position of the trimmed content on the stock paper, i.e. odd pages will have content on the right side instead of the left and even pages will have content on the left side instead of the right.
\setlength{\trimedge}{\stockwidth}
\addtolength{\trimedge}{-\paperwidth}

% To bring content to center.
\addtolength{\trimtop}{2.85cm}
% To bring content to center.
\addtolength{\trimedge}{-2cm}

% Display other style of trim marks.
\quarkmarks

% Put jobname in left top trim mark.
\renewcommand*{\tmarktl}{\registrationColour{%
  \begin{picture}(0,0)
    \setlength{\unitlength}{1bp}\thicklines
    \put(-36,0){\line(1,0){24}}
    \put(0,12){\line(0,1){24}}
    \put(3,27){\normalfont\ttfamily\fontsize{8bp}{10bp}\selectfont\jobname\ \
      \today\ \ \printtime\ \ Sheet \thesheetsequence}
  \end{picture}}}


% Makes sure your specifications are correct and implements them in the document.
\checkandfixthelayout
}%

\def\A4paperUL30BR30-A{%
\setstocksize{297mm}{210mm}
\settrimmedsize{297mm}{210mm}{*}
\setpagecc{\paperheight}{\paperwidth}{*}
\setlrmarginsandblock{30mm}{20mm}{*}
\setulmarginsandblock{30mm}{20mm}{*}
\checkandfixthelayout
}%
%\usepackage{fontspec}
% Há algum tempo havia, na documentação do fontspec, o aviso
% para carregá-la antes da polyglossia. Continuo fazendo assim...
\usepackage{polyglossia}


% Definição de opções para fonte(s) utilizada(s)
\setmainfont{Arial}
	% Se possuir a fonte Arial instalada no seu sistema op.,
	% com o nome Arial, pode chamá-la assim que o fontspec
	% a colocará como fonte principal do texto.

\import{preambulo/}{dimensoes}
\import{preambulo/}{fontes}
% Aviso: dependendo das fontes que for usar no arquivo 'geral',
% pode ser que estes tenha de vir antes de pacotes chamados no
% arquivo 'fontes'. Esteja atento para tais situações e tomar
% as medidas necessárias.
%\usepackage{blindtext} 
%Pacote para gerar texto aleatório

\usepackage{amsmath,amsthm,amssymb,amsfonts,amscd}
% Vários pacotes matemáticos

\import{preambulo/}{geral}

\A4paperUL30BR30-A
% Comando definido no arquivo 'dimensoes' para tamanho de papel,
% margens, bloco de texto, etc

\begin{document}

%\blindmathtrue
%\blindtext[5]
%\blinditemize
%\blindenumerate
%\blinddescription
%\blindmathpaper

%\chapter{Título 1}
Lorem ipsum dolor sit amet, consectetur adipiscing elit. Curabitur at mi vel tortor luctus ullamcorper. Nullam velit mauris, fermentum vitae blandit vehicula, dapibus eu risus. Fusce eleifend mollis ultrices. Duis ac magna urna. Quisque faucibus nulla nisl. Suspendisse a erat in erat tincidunt fringilla. Praesent mollis quam turpis, eget auctor elit dignissim accumsan. Vestibulum ante ipsum primis in faucibus orci luctus et ultrices posuere cubilia Curae; Quisque pharetra consectetur ligula quis laoreet. In et cursus dolor. Sed a sollicitudin tellus. Praesent ac risus quis justo suscipit ullamcorper. Curabitur augue est, placerat sed ligula sed, interdum porttitor leo. Etiam nec leo magna.

Nunc mollis elit at laoreet varius. Ut congue, neque eget placerat interdum, velit nulla mattis elit, vitae interdum nisi risus nec magna. Aenean sed lectus iaculis, rutrum turpis id, varius massa. Nam pharetra metus lorem, et sollicitudin urna ultricies at. Etiam condimentum justo sed malesuada mollis. Nunc eget leo aliquam, bibendum magna ut, lobortis sem. Sed ut metus elementum, tempor neque a, laoreet leo. Suspendisse lobortis mattis nisi id luctus.

Integer id nisl tristique, eleifend sapien condimentum, lobortis tellus. Nulla suscipit orci sed pretium vestibulum. Aenean tristique eget orci sed posuere. Maecenas lacus odio, porta ac nibh sed, pretium commodo massa. Morbi ac odio tortor. Ut at quam eu enim mattis consequat. Fusce ut molestie tortor, sed convallis augue. Donec in augue dui. Aenean et porta nibh. Suspendisse tempus vehicula quam non sodales. Nunc rutrum tellus in augue ultricies euismod. Vestibulum luctus lacinia pulvinar.
%\section{Seção 1}

Lorem ipsum dolor sit amet, consectetur adipiscing elit. Curabitur at mi vel tortor luctus ullamcorper. Nullam velit mauris, fermentum vitae blandit vehicula, dapibus eu risus. Fusce eleifend mollis ultrices. Duis ac magna urna. Quisque faucibus nulla nisl. Suspendisse a erat in erat tincidunt fringilla. Praesent mollis quam turpis, eget auctor elit dignissim accumsan. Vestibulum ante ipsum primis in faucibus orci luctus et ultrices posuere cubilia Curae; Quisque pharetra consectetur ligula quis laoreet. In et cursus dolor. Sed a sollicitudin tellus. Praesent ac risus quis justo suscipit ullamcorper. Curabitur augue est, placerat sed ligula sed, interdum porttitor leo. Etiam nec leo magna.

Nunc mollis elit at laoreet varius. Ut congue, neque eget placerat interdum, velit nulla mattis elit, vitae interdum nisi risus nec magna. Aenean sed lectus iaculis, rutrum turpis id, varius massa. Nam pharetra metus lorem, et sollicitudin urna ultricies at. Etiam condimentum justo sed malesuada mollis. Nunc eget leo aliquam, bibendum magna ut, lobortis sem. Sed ut metus elementum, tempor neque a, laoreet leo. Suspendisse lobortis mattis nisi id luctus.

Integer id nisl tristique, eleifend sapien condimentum, lobortis tellus. Nulla suscipit orci sed pretium vestibulum. Aenean tristique eget orci sed posuere. Maecenas lacus odio, porta ac nibh sed, pretium commodo massa. Morbi ac odio tortor. Ut at quam eu enim mattis consequat. Fusce ut molestie tortor, sed convallis augue. Donec in augue dui. Aenean et porta nibh. Suspendisse tempus vehicula quam non sodales. Nunc rutrum tellus in augue ultricies euismod. Vestibulum luctus lacinia pulvinar.

Vivamus quis malesuada nibh. Integer finibus diam euismod, placerat justo vel, ullamcorper lorem. Mauris vitae tincidunt nulla, nec faucibus diam. Maecenas porttitor diam magna, vel condimentum sapien elementum id. Pellentesque in lectus aliquet, ornare neque ac, dictum dolor. Nulla quis nibh vitae ligula varius mattis eget sed nisl. Praesent quis odio magna. Phasellus at lacinia risus, consequat gravida est. Quisque ullamcorper arcu eu lacinia pulvinar. In quis nisi pretium, fringilla mi et, tristique augue.

Nam tortor justo, ornare eu arcu non, dictum sodales lacus. Sed ut vehicula arcu. Mauris non faucibus metus, in porttitor lacus. Suspendisse mattis turpis sed vulputate egestas. Vivamus eros nisl, congue eget vehicula vitae, cursus sit amet nunc. Fusce vitae libero nec justo condimentum posuere. Sed bibendum lectus eget mollis sollicitudin. In hac habitasse platea dictumst. Ut eget urna ut mauris pharetra dictum eget eget ex. Curabitur aliquet arcu ut egestas varius. Aliquam sed interdum nulla. Sed lectus risus, elementum at rhoncus eu, ullamcorper quis est. Mauris et massa convallis dolor finibus dictum non eget ipsum. Aliquam dictum lacus risus, eu mollis turpis posuere ac. Sed ut nibh vitae turpis blandit dapibus.
%\section{Seção 1}

Lorem ipsum dolor sit amet, consectetur adipiscing elit. Curabitur at mi vel tortor luctus ullamcorper. Nullam velit mauris, fermentum vitae blandit vehicula, dapibus eu risus. Fusce eleifend mollis ultrices. Duis ac magna urna. Quisque faucibus nulla nisl. Suspendisse a erat in erat tincidunt fringilla. Praesent mollis quam turpis, eget auctor elit dignissim accumsan. Vestibulum ante ipsum primis in faucibus orci luctus et ultrices posuere cubilia Curae; Quisque pharetra consectetur ligula quis laoreet. In et cursus dolor. Sed a sollicitudin tellus. Praesent ac risus quis justo suscipit ullamcorper. Curabitur augue est, placerat sed ligula sed, interdum porttitor leo. Etiam nec leo magna.

Nunc mollis elit at laoreet varius. Ut congue, neque eget placerat interdum, velit nulla mattis elit, vitae interdum nisi risus nec magna. Aenean sed lectus iaculis, rutrum turpis id, varius massa. Nam pharetra metus lorem, et sollicitudin urna ultricies at. Etiam condimentum justo sed malesuada mollis. Nunc eget leo aliquam, bibendum magna ut, lobortis sem. Sed ut metus elementum, tempor neque a, laoreet leo. Suspendisse lobortis mattis nisi id luctus.

Integer id nisl tristique, eleifend sapien condimentum, lobortis tellus. Nulla suscipit orci sed pretium vestibulum. Aenean tristique eget orci sed posuere. Maecenas lacus odio, porta ac nibh sed, pretium commodo massa. Morbi ac odio tortor. Ut at quam eu enim mattis consequat. Fusce ut molestie tortor, sed convallis augue. Donec in augue dui. Aenean et porta nibh. Suspendisse tempus vehicula quam non sodales. Nunc rutrum tellus in augue ultricies euismod. Vestibulum luctus lacinia pulvinar.

Vivamus quis malesuada nibh. Integer finibus diam euismod, placerat justo vel, ullamcorper lorem. Mauris vitae tincidunt nulla, nec faucibus diam. Maecenas porttitor diam magna, vel condimentum sapien elementum id. Pellentesque in lectus aliquet, ornare neque ac, dictum dolor. Nulla quis nibh vitae ligula varius mattis eget sed nisl. Praesent quis odio magna. Phasellus at lacinia risus, consequat gravida est. Quisque ullamcorper arcu eu lacinia pulvinar. In quis nisi pretium, fringilla mi et, tristique augue.

Nam tortor justo, ornare eu arcu non, dictum sodales lacus. Sed ut vehicula arcu. Mauris non faucibus metus, in porttitor lacus. Suspendisse mattis turpis sed vulputate egestas. Vivamus eros nisl, congue eget vehicula vitae, cursus sit amet nunc. Fusce vitae libero nec justo condimentum posuere. Sed bibendum lectus eget mollis sollicitudin. In hac habitasse platea dictumst. Ut eget urna ut mauris pharetra dictum eget eget ex. Curabitur aliquet arcu ut egestas varius. Aliquam sed interdum nulla. Sed lectus risus, elementum at rhoncus eu, ullamcorper quis est. Mauris et massa convallis dolor finibus dictum non eget ipsum. Aliquam dictum lacus risus, eu mollis turpis posuere ac. Sed ut nibh vitae turpis blandit dapibus.
%\chapter{Título 2}
Lorem ipsum dolor sit amet, consectetur adipiscing elit. Curabitur at mi vel tortor luctus ullamcorper. Nullam velit mauris, fermentum vitae blandit vehicula, dapibus eu risus. Fusce eleifend mollis ultrices. Duis ac magna urna. Quisque faucibus nulla nisl. Suspendisse a erat in erat tincidunt fringilla. Praesent mollis quam turpis, eget auctor elit dignissim accumsan. Vestibulum ante ipsum primis in faucibus orci luctus et ultrices posuere cubilia Curae; Quisque pharetra consectetur ligula quis laoreet. In et cursus dolor. Sed a sollicitudin tellus. Praesent ac risus quis justo suscipit ullamcorper. Curabitur augue est, placerat sed ligula sed, interdum porttitor leo. Etiam nec leo magna.

Nunc mollis elit at laoreet varius. Ut congue, neque eget placerat interdum, velit nulla mattis elit, vitae interdum nisi risus nec magna. Aenean sed lectus iaculis, rutrum turpis id, varius massa. Nam pharetra metus lorem, et sollicitudin urna ultricies at. Etiam condimentum justo sed malesuada mollis. Nunc eget leo aliquam, bibendum magna ut, lobortis sem. Sed ut metus elementum, tempor neque a, laoreet leo. Suspendisse lobortis mattis nisi id luctus.

Integer id nisl tristique, eleifend sapien condimentum, lobortis tellus. Nulla suscipit orci sed pretium vestibulum. Aenean tristique eget orci sed posuere. Maecenas lacus odio, porta ac nibh sed, pretium commodo massa. Morbi ac odio tortor. Ut at quam eu enim mattis consequat. Fusce ut molestie tortor, sed convallis augue. Donec in augue dui. Aenean et porta nibh. Suspendisse tempus vehicula quam non sodales. Nunc rutrum tellus in augue ultricies euismod. Vestibulum luctus lacinia pulvinar.

\subimport{./}{1.1.tex}

%\chapter{Título 1}
Lorem ipsum dolor sit amet, consectetur adipiscing elit. Curabitur at mi vel tortor luctus ullamcorper. Nullam velit mauris, fermentum vitae blandit vehicula, dapibus eu risus. Fusce eleifend mollis ultrices. Duis ac magna urna. Quisque faucibus nulla nisl. Suspendisse a erat in erat tincidunt fringilla. Praesent mollis quam turpis, eget auctor elit dignissim accumsan. Vestibulum ante ipsum primis in faucibus orci luctus et ultrices posuere cubilia Curae; Quisque pharetra consectetur ligula quis laoreet. In et cursus dolor. Sed a sollicitudin tellus. Praesent ac risus quis justo suscipit ullamcorper. Curabitur augue est, placerat sed ligula sed, interdum porttitor leo. Etiam nec leo magna.

Nunc mollis elit at laoreet varius. Ut congue, neque eget placerat interdum, velit nulla mattis elit, vitae interdum nisi risus nec magna. Aenean sed lectus iaculis, rutrum turpis id, varius massa. Nam pharetra metus lorem, et sollicitudin urna ultricies at. Etiam condimentum justo sed malesuada mollis. Nunc eget leo aliquam, bibendum magna ut, lobortis sem. Sed ut metus elementum, tempor neque a, laoreet leo. Suspendisse lobortis mattis nisi id luctus.

Integer id nisl tristique, eleifend sapien condimentum, lobortis tellus. Nulla suscipit orci sed pretium vestibulum. Aenean tristique eget orci sed posuere. Maecenas lacus odio, porta ac nibh sed, pretium commodo massa. Morbi ac odio tortor. Ut at quam eu enim mattis consequat. Fusce ut molestie tortor, sed convallis augue. Donec in augue dui. Aenean et porta nibh. Suspendisse tempus vehicula quam non sodales. Nunc rutrum tellus in augue ultricies euismod. Vestibulum luctus lacinia pulvinar.
%\section{Seção 1}

Lorem ipsum dolor sit amet, consectetur adipiscing elit. Curabitur at mi vel tortor luctus ullamcorper. Nullam velit mauris, fermentum vitae blandit vehicula, dapibus eu risus. Fusce eleifend mollis ultrices. Duis ac magna urna. Quisque faucibus nulla nisl. Suspendisse a erat in erat tincidunt fringilla. Praesent mollis quam turpis, eget auctor elit dignissim accumsan. Vestibulum ante ipsum primis in faucibus orci luctus et ultrices posuere cubilia Curae; Quisque pharetra consectetur ligula quis laoreet. In et cursus dolor. Sed a sollicitudin tellus. Praesent ac risus quis justo suscipit ullamcorper. Curabitur augue est, placerat sed ligula sed, interdum porttitor leo. Etiam nec leo magna.

Nunc mollis elit at laoreet varius. Ut congue, neque eget placerat interdum, velit nulla mattis elit, vitae interdum nisi risus nec magna. Aenean sed lectus iaculis, rutrum turpis id, varius massa. Nam pharetra metus lorem, et sollicitudin urna ultricies at. Etiam condimentum justo sed malesuada mollis. Nunc eget leo aliquam, bibendum magna ut, lobortis sem. Sed ut metus elementum, tempor neque a, laoreet leo. Suspendisse lobortis mattis nisi id luctus.

Integer id nisl tristique, eleifend sapien condimentum, lobortis tellus. Nulla suscipit orci sed pretium vestibulum. Aenean tristique eget orci sed posuere. Maecenas lacus odio, porta ac nibh sed, pretium commodo massa. Morbi ac odio tortor. Ut at quam eu enim mattis consequat. Fusce ut molestie tortor, sed convallis augue. Donec in augue dui. Aenean et porta nibh. Suspendisse tempus vehicula quam non sodales. Nunc rutrum tellus in augue ultricies euismod. Vestibulum luctus lacinia pulvinar.

Vivamus quis malesuada nibh. Integer finibus diam euismod, placerat justo vel, ullamcorper lorem. Mauris vitae tincidunt nulla, nec faucibus diam. Maecenas porttitor diam magna, vel condimentum sapien elementum id. Pellentesque in lectus aliquet, ornare neque ac, dictum dolor. Nulla quis nibh vitae ligula varius mattis eget sed nisl. Praesent quis odio magna. Phasellus at lacinia risus, consequat gravida est. Quisque ullamcorper arcu eu lacinia pulvinar. In quis nisi pretium, fringilla mi et, tristique augue.

Nam tortor justo, ornare eu arcu non, dictum sodales lacus. Sed ut vehicula arcu. Mauris non faucibus metus, in porttitor lacus. Suspendisse mattis turpis sed vulputate egestas. Vivamus eros nisl, congue eget vehicula vitae, cursus sit amet nunc. Fusce vitae libero nec justo condimentum posuere. Sed bibendum lectus eget mollis sollicitudin. In hac habitasse platea dictumst. Ut eget urna ut mauris pharetra dictum eget eget ex. Curabitur aliquet arcu ut egestas varius. Aliquam sed interdum nulla. Sed lectus risus, elementum at rhoncus eu, ullamcorper quis est. Mauris et massa convallis dolor finibus dictum non eget ipsum. Aliquam dictum lacus risus, eu mollis turpis posuere ac. Sed ut nibh vitae turpis blandit dapibus.
%\section{Seção 1}

Lorem ipsum dolor sit amet, consectetur adipiscing elit. Curabitur at mi vel tortor luctus ullamcorper. Nullam velit mauris, fermentum vitae blandit vehicula, dapibus eu risus. Fusce eleifend mollis ultrices. Duis ac magna urna. Quisque faucibus nulla nisl. Suspendisse a erat in erat tincidunt fringilla. Praesent mollis quam turpis, eget auctor elit dignissim accumsan. Vestibulum ante ipsum primis in faucibus orci luctus et ultrices posuere cubilia Curae; Quisque pharetra consectetur ligula quis laoreet. In et cursus dolor. Sed a sollicitudin tellus. Praesent ac risus quis justo suscipit ullamcorper. Curabitur augue est, placerat sed ligula sed, interdum porttitor leo. Etiam nec leo magna.

Nunc mollis elit at laoreet varius. Ut congue, neque eget placerat interdum, velit nulla mattis elit, vitae interdum nisi risus nec magna. Aenean sed lectus iaculis, rutrum turpis id, varius massa. Nam pharetra metus lorem, et sollicitudin urna ultricies at. Etiam condimentum justo sed malesuada mollis. Nunc eget leo aliquam, bibendum magna ut, lobortis sem. Sed ut metus elementum, tempor neque a, laoreet leo. Suspendisse lobortis mattis nisi id luctus.

Integer id nisl tristique, eleifend sapien condimentum, lobortis tellus. Nulla suscipit orci sed pretium vestibulum. Aenean tristique eget orci sed posuere. Maecenas lacus odio, porta ac nibh sed, pretium commodo massa. Morbi ac odio tortor. Ut at quam eu enim mattis consequat. Fusce ut molestie tortor, sed convallis augue. Donec in augue dui. Aenean et porta nibh. Suspendisse tempus vehicula quam non sodales. Nunc rutrum tellus in augue ultricies euismod. Vestibulum luctus lacinia pulvinar.

Vivamus quis malesuada nibh. Integer finibus diam euismod, placerat justo vel, ullamcorper lorem. Mauris vitae tincidunt nulla, nec faucibus diam. Maecenas porttitor diam magna, vel condimentum sapien elementum id. Pellentesque in lectus aliquet, ornare neque ac, dictum dolor. Nulla quis nibh vitae ligula varius mattis eget sed nisl. Praesent quis odio magna. Phasellus at lacinia risus, consequat gravida est. Quisque ullamcorper arcu eu lacinia pulvinar. In quis nisi pretium, fringilla mi et, tristique augue.

Nam tortor justo, ornare eu arcu non, dictum sodales lacus. Sed ut vehicula arcu. Mauris non faucibus metus, in porttitor lacus. Suspendisse mattis turpis sed vulputate egestas. Vivamus eros nisl, congue eget vehicula vitae, cursus sit amet nunc. Fusce vitae libero nec justo condimentum posuere. Sed bibendum lectus eget mollis sollicitudin. In hac habitasse platea dictumst. Ut eget urna ut mauris pharetra dictum eget eget ex. Curabitur aliquet arcu ut egestas varius. Aliquam sed interdum nulla. Sed lectus risus, elementum at rhoncus eu, ullamcorper quis est. Mauris et massa convallis dolor finibus dictum non eget ipsum. Aliquam dictum lacus risus, eu mollis turpis posuere ac. Sed ut nibh vitae turpis blandit dapibus.
%\chapter{Título 2}
Lorem ipsum dolor sit amet, consectetur adipiscing elit. Curabitur at mi vel tortor luctus ullamcorper. Nullam velit mauris, fermentum vitae blandit vehicula, dapibus eu risus. Fusce eleifend mollis ultrices. Duis ac magna urna. Quisque faucibus nulla nisl. Suspendisse a erat in erat tincidunt fringilla. Praesent mollis quam turpis, eget auctor elit dignissim accumsan. Vestibulum ante ipsum primis in faucibus orci luctus et ultrices posuere cubilia Curae; Quisque pharetra consectetur ligula quis laoreet. In et cursus dolor. Sed a sollicitudin tellus. Praesent ac risus quis justo suscipit ullamcorper. Curabitur augue est, placerat sed ligula sed, interdum porttitor leo. Etiam nec leo magna.

Nunc mollis elit at laoreet varius. Ut congue, neque eget placerat interdum, velit nulla mattis elit, vitae interdum nisi risus nec magna. Aenean sed lectus iaculis, rutrum turpis id, varius massa. Nam pharetra metus lorem, et sollicitudin urna ultricies at. Etiam condimentum justo sed malesuada mollis. Nunc eget leo aliquam, bibendum magna ut, lobortis sem. Sed ut metus elementum, tempor neque a, laoreet leo. Suspendisse lobortis mattis nisi id luctus.

Integer id nisl tristique, eleifend sapien condimentum, lobortis tellus. Nulla suscipit orci sed pretium vestibulum. Aenean tristique eget orci sed posuere. Maecenas lacus odio, porta ac nibh sed, pretium commodo massa. Morbi ac odio tortor. Ut at quam eu enim mattis consequat. Fusce ut molestie tortor, sed convallis augue. Donec in augue dui. Aenean et porta nibh. Suspendisse tempus vehicula quam non sodales. Nunc rutrum tellus in augue ultricies euismod. Vestibulum luctus lacinia pulvinar.

\subimport{./}{1.1.tex}

%\import{documentos/}{1}
%\import{documentos/}{1.1}
%\import{documentos/}{1.2}
%\import{documentos/}{2}

\import{documentos/}{1a}
\import{documentos/}{2}

\end{document}
